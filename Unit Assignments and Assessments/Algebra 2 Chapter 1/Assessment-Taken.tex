\documentclass[12pt]{exam}

\usepackage{amsmath}
\usepackage{amssymb}
\usepackage{amsfonts}
\usepackage{cancel}
\usepackage{pifont}
\usepackage{graphicx}

% Header and Footer
\pagestyle{headandfoot}
\firstpageheader{8:42 - 9:09 and 9:16 - 9:24 \\ Integrated Algebra 2 and Precalculus}{}{35 minutes \\ Exam: Chapter 1 of Algebra 2}
\runningheader{Integrated Algebra 2 and Precalculus}{}{Exam: Chapter 1 of Algebra 2}
\firstpagefooter{}{Page \thepage\ of \numpages}{}
\runningfooter{}{Page \thepage\ of \numpages}{}

% Title
\newcommand{\examtitle}{Basic Concepts of Algebra}

% Instructions
\newcommand{\instructions}{
    \noindent\rule{\textwidth}{0.5pt}
    \begin{center}
    \textbf{Instructions:} Answer all questions to the best of your ability. Show all your work in the space provided for full credit.
    \end{center}
    \noindent\rule{\textwidth}{0.5pt}
}

\begin{document}

\begin{center}
\textbf{\Large \examtitle} \\
\vspace{0.5cm}
\makebox[0.4\textwidth]{Name: \underline{Krish Arora \hspace{3.5cm}} \enspace}
\hspace{0.1\textwidth}
\makebox[0.4\textwidth]{Date: \underline{7/8/2025 \hspace{3.5cm}} \enspace}
\end{center}

\instructions
\vspace{0.5cm}

\begin{questions}

%--------------------------------------------------------------------------
% Section 1: Evaluating and Simplifying Expressions
%--------------------------------------------------------------------------

\pointsinrightmargin
\question[16]
Evaluate each of the following expressions when $a = -8$ and $b = \frac{1}{2}$.

\begin{parts}
    \part $ab + 2b + 3a$

    \begin{minipage}[t][2cm][t]{\linewidth}
        $\displaystyle (-8)(\frac{1}{2}) + 2(\frac{1}{2}) + 3(-8) = -4 + 1 - 24 = \boxed{-27}$
    \end{minipage}
    
    \part $\dfrac{a}{b}$

    \begin{minipage}[t][2cm][t]{\linewidth}
        $\displaystyle (-8) \div \frac{1}{2} = \boxed{-16}$
    \end{minipage}

    \part $4(a^2b + b^2a)$

    \begin{minipage}[t][3cm][t]{\linewidth}
        $\displaystyle 4((-8)^2 \cdot (0.5) + (0.5)^2 \cdot (-8)) = 4(32-2) = \boxed{120}$
    \end{minipage}

    \part $(a - 2)\sqrt{-ab}$

    \begin{minipage}[t][3cm][t]{\linewidth}
        $\displaystyle ((-8)-2) \sqrt{-(-8)} \sqrt{0.5} = -10 \cdot \sqrt{8} \cdot \frac{1}{\sqrt{2}} = -10 \cdot 2 \cdot \sqrt{2} \cdot \frac{1}{\sqrt{2}} = \boxed{-20}$
    \end{minipage}
\end{parts}

\question[5]
Simplify the product as much as possible.
\[ \frac{2}{3a^2 - 6b} \cdot \frac{9a^3 - 18ab}{10a^2} \]
\begin{minipage}[t][4cm][t]{\linewidth}
    $\displaystyle \frac{2}{3(a^2-2b)} \cdot \frac{9a(a^2-2b)}{10a^2}$
    \\[10pt] $\displaystyle \frac{\cancel{2} \ 1}{3\cancel{(a^2-2b)}} \cdot \frac{9a\cancel{(a^2-2b)}}{\cancel{10} \ 5a^2}$
    \\[10pt] $\displaystyle \frac{9a}{15a^2} = \boxed{\frac{3}{5a}}$
\end{minipage}

%--------------------------------------------------------------------------
% Section 2: Properties of Number Systems
%--------------------------------------------------------------------------

\question[12]
Decide whether each set is a field under the operations of addition and multiplication. If the set is not a field, name at least one field property that does not hold.

\begin{parts}
    \part The natural numbers ($\mathbb{N}$) \text{\huge{\ding{55}}}

    \begin{minipage}[t][2.5cm][t]{\linewidth}
        No Multiplicative Inverse
    \end{minipage}

    \part The integers ($\mathbb{Z}$) \text{\huge{\ding{55}}}

    \begin{minipage}[t][2.5cm][t]{\linewidth}
        No Multiplicative Inverse
    \end{minipage}

    \part The rational numbers ($\mathbb{Q}$) \text{\huge{\ding{51}}}

    \begin{minipage}[t][2.5cm][t]{\linewidth}
        Closed Under Addition \& Multiplication
        $\\[8pt]$ Has Inverses
        $\\[8pt]$ Distributive and Associative Property
    \end{minipage}

    \part The negative rational numbers \text{\huge{\ding{55}}}

    \begin{minipage}[t][2.5cm][t]{\linewidth}
        Not Closed Under Multiplication: $-3 \cdot (-2) = 6$ (6 not in set).
    \end{minipage}
\end{parts}

%--------------------------------------------------------------------------
% Section 3: Solving Equations
%--------------------------------------------------------------------------

\question[10]
Solve for the variable in each equation.

\begin{parts}
    \part Find all values of $y$ such that $\dfrac{3}{2 + \sqrt{y}} + \dfrac{4}{2 + \sqrt{y}} = 1$.

    \begin{minipage}[t][5cm][t]{\linewidth}
        $\displaystyle \frac{7}{2 + \sqrt{y}} = 1
        \\[10pt] 2 + \sqrt{y} = 7
        \\[10pt] \sqrt{y} = 5
        \\[10pt] \boxed{y = 25}$
    \end{minipage}

    \part What values of $x$ satisfy $\dfrac{\sqrt{x+1} + \sqrt{x-1}}{\sqrt{x+1} - \sqrt{x-1}} = 3$?

    \begin{minipage}[t][5cm][t]{\linewidth}
        $\displaystyle \sqrt{x+1} + \sqrt{x-1} = 3\sqrt{x+1} - 3\sqrt{x-1}
        \\[8pt] 4\sqrt{x-1} = 2\sqrt{x+1}
        \\[8pt] \sqrt{x+1} = 2\sqrt{x-1}
        \\[8pt] \sqrt{x+1} = \sqrt{4}\sqrt{x-1}
        \\[8pt] x+1 = 4x-4
        \\[8pt] 3x = 5 ; \boxed{x = \frac{5}{3}}$
    \end{minipage}
\end{parts}

%--------------------------------------------------------------------------
% Section 4: Problem Solving
%--------------------------------------------------------------------------

\question[8]
\begin{parts}
    \part Let $x$ be the middle integer of three consecutive integers. What is the sum of these three integers in terms of $x$?

    \begin{minipage}[t][3cm][t]{\linewidth}
        $\displaystyle (x-1) + x + (x+1) = \boxed{3x}$
    \end{minipage}
    
    \part The sum of 23 consecutive integers is 2323. What is the largest of the integers? (Hint: Use the result from the first part.)

    \begin{minipage}[t][4cm][t]{\linewidth}
        $\displaystyle (x-11) + (x-10) + (x-9)... + x + ...(x+9) + (x+10) + (x+11) = 23x = 2323
        \\[10pt] 2323/23 = 101
        \\[10pt] \boxed{x = 101}$
    \end{minipage}
\end{parts}

\question[6]
A grocer wants to mix peanuts and cashews to produce 20 lb of mixed nuts worth \$6.20/lb. How many pounds of each kind of nut should she use if peanuts cost \$4.80/lb and cashews cost \$8.00/lb?

\begin{minipage}[t][6cm][t]{\linewidth}
    $\displaystyle 4.8x + 8(20-x) = 20(6.2)
    \\[10pt] 160 - 3.2x = 124
    \\[10pt] 3.2x = 36
    \\[10pt] 16x = 180
    \\[10pt] \boxed{x = \frac{45}{4}}$
\end{minipage}

\end{questions}

\end{document}