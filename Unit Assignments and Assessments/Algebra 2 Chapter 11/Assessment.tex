\documentclass[12pt]{exam}

\usepackage{amsmath}
\usepackage{amssymb}
\usepackage{amsfonts}
\usepackage{graphicx}
\usepackage{tikz}
\usetikzlibrary{arrows.meta}

% Header and Footer
\pagestyle{headandfoot}
\firstpageheader{Integrated Algebra 2 and Precalculus}{}{Exam: Chapter 11 of Algebra 2}
\runningheader{Integrated Algebra 2 and Precalculus}{}{Exam: Chapter 11 of Algebra 2}
\firstpagefooter{}{Page \thepage\ of \numpages}{}
\runningfooter{}{Page \thepage\ of \numpages}{}

% Title
\newcommand{\examtitle}{Sequences and Series}

% Instructions
\newcommand{\instructions}{
    \noindent\rule{\textwidth}{0.5pt}
    \begin{center}
    \textbf{Instructions:} Answer all questions to the best of your ability. Show all your work in the space provided for full credit.
    \end{center}
    \noindent\rule{\textwidth}{0.5pt}
}

\begin{document}

\begin{center}
\textbf{\Large \examtitle} \\
\vspace{0.5cm}
\makebox[0.4\textwidth]{Name: \enspace\hrulefill}
\hspace{0.1\textwidth}
\makebox[0.4\textwidth]{Date: \enspace\hrulefill}
\end{center}

\instructions
\vspace{0.5cm}

\begin{questions}

\pointsinrightmargin

\question[8]
If the fourth term of an arithmetic sequence is 200 and the eighth term is 500, what is the sixth term?
\vspace*{4cm}

\question[8]
If the fourth term of a geometric sequence of positive numbers is 200 and the eighth term is 800, what is the sixth term?
\vspace*{4cm}

\question[8]
A geometric sequence has common ratio $r$, where $r \neq 0$, and the $n^{\text{th}}$ term is $b$. Find an expression for the first term of the sequence in terms of $r, n$, and $b$.
\vspace*{4cm}

\newpage

\question[10]
An infinite geometric series has common ratio $-\frac{1}{2}$ and sum 45. What is the first term of the series?
\vspace*{4cm}

\question[12]
(a) What is the sum of the first 50 positive integers?
\vspace*{3cm}
(b) The sum of the first $k$ positive integers is 990. What is $k$?
\vspace*{3cm}

\newpage

\question[12]
(a) Evaluate $\sum_{i=1}^{10} (2i - 5)$.
\vspace*{3cm}

(b) Evaluate $\sum_{i=1}^{72} 5$.
\vspace*{3cm}

(c) Evaluate $\sum_{i=1}^{7} 3^i$.
\vspace*{3cm}

\newpage

\question[12]
If the sum of the first $3n$ positive integers is 150 more than the sum of the first $n$ positive integers, then what is the sum of the first $4n$ positive integers?
\vspace*{5cm}

\question[12]
In this problem we evaluate the series
\[ \frac{1}{1 \cdot 3} + \frac{1}{2 \cdot 4} + \frac{1}{3 \cdot 5} + \dots + \frac{1}{98 \cdot 100} \]
(a) Notice that each fraction in the sum has the form $\frac{1}{n(n+2)}$ for some positive integer $n$. Find constants $A$ and $B$ such that
\[ \frac{1}{n(n+2)} = \frac{A}{n} + \frac{B}{n+2} \]
\vspace*{4cm}
(b) Use your answer to part (a) to find the desired sum.
\vspace*{4cm}

\newpage

\question[12]
Evaluate the sum $\frac{1}{2^1} + \frac{2}{2^2} + \frac{3}{2^3} + \frac{4}{2^4} + \dots$
\vspace*{5cm}

\newpage

\question[16]
In this problem we derive a formula for the sum of the first $n$ perfect squares. Let $n$ be a positive integer, and let $S = 1^2 + 2^2 + 3^2 + \dots + n^2$.

(a) Prove that $1 + 3 + 5 + \dots + (2k-1) = k^2$.
\vspace*{4cm}

(b) Use part (a) to show that $S = (1)(n) + (3)(n-1) + (5)(n-2) + \dots + (2n-1)(1)$
\textit{Hint: If we write each square as the sum of odd numbers as described in part (a), for how many of the $n$ squares will 1 be among the odd numbers in the sum? For how many of them will 3 be among the odd numbers in the sum?}
\vspace*{4cm}

(c) Show that $2S = (2)(1) + (4)(2) + (6)(3) + \dots + (2n)(n)$
\vspace*{4cm}

(d) Add the equations in part (b) and (c) to conclude that $S = \frac{n(n+1)(2n+1)}{6}$.
\textit{Hint: Find a clever way to combine each term from the series in part (a) with a term in part (b).}
\vspace*{4cm}


\end{questions}

\end{document} 