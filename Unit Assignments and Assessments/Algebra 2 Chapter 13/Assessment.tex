\documentclass[12pt]{exam}

\usepackage{amsmath}
\usepackage{amssymb}
\usepackage{amsfonts}
\usepackage{graphicx}
\usepackage{tikz}
\usetikzlibrary{arrows.meta}

% Header and Footer
\pagestyle{headandfoot}
\firstpageheader{Integrated Algebra 2 and Precalculus}{}{Exam: Chapter 13 of Algebra 2}
\runningheader{Integrated Algebra 2 and Precalculus}{}{Exam: Chapter 13 of Algebra 2}
\firstpagefooter{}{Page \thepage\ of \numpages}{}
\runningfooter{}{Page \thepage\ of \numpages}{}

% Title
\newcommand{\examtitle}{Trigonometric Graphs and Identities}

% Instructions
\newcommand{\instructions}{
    \noindent\rule{\textwidth}{0.5pt}
    \begin{center}
    \textbf{Instructions:} Answer all questions to the best of your ability. Show all your work in the space provided for full credit.
    \end{center}
    \noindent\rule{\textwidth}{0.5pt}
}

\begin{document}

\begin{center}
\textbf{\Large \examtitle} \\
\vspace{0.5cm}
\makebox[0.4\textwidth]{Name: \enspace\hrulefill}
\hspace{0.1\textwidth}
\makebox[0.4\textwidth]{Date: \enspace\hrulefill}
\end{center}

\instructions
\vspace{0.5cm}

\begin{questions}

\pointsinrightmargin

\question[10]
Convert the following angle measures as indicated:
\begin{parts}
    \part Convert $\frac{2\pi}{3}$ radians to degrees.
    \vspace*{2cm}
    \part Convert $150^\circ$ to radians.
    \vspace*{2cm}
    \part Convert $\frac{5\pi}{6}$ radians to degrees.
    \vspace*{2cm}
    \part Convert $240^\circ$ to radians.
    \vspace*{2cm}
\end{parts}

\question[10]
The graph of a sinusoidal function of the form $y = a \cos(b(x-c)) + d$ has a maximum point at $(\pi/3, 5)$ and a subsequent minimum point at $(\pi, 1)$. Find the values for $a, b, c,$ and $d$, assuming $a>0$, $b>0$, and $c$ is the smallest possible positive value.
\vspace*{5cm}

\newpage

\question[10]
Determine the equations of all vertical asymptotes for the function $f(x) = 2 \sec(3x - \frac{\pi}{2})$ on the interval $[0, 2\pi]$.
\vspace*{5cm}

\question[10]
Prove the following trigonometric identity:
\[ \frac{\cos A - \sin A + 1}{\cos A + \sin A - 1} = \csc A + \cot A \]
\vspace*{6cm}

\newpage

\question[10]
Given that $\sin \alpha = \frac{4}{5}$ with $\alpha$ in Quadrant II, and $\cos \beta = \frac{5}{13}$ with $\beta$ in Quadrant IV, find the exact value of $\cos(\alpha - \beta)$.
\vspace*{5cm}

\question[10]
Solve the equation $\cos(2x) + 3\sin x - 2 = 0$ for all values of $x$ in the interval $0 \le x < 2\pi$.
\vspace*{5cm}

\newpage

\question[10]
Use a half-angle formula to find the exact value of $\tan(105^\circ)$.
\vspace*{5cm}

\question[10]
Prove the identity $\tan(4\theta) = \frac{4\tan\theta - 4\tan^3\theta}{1 - 6\tan^2\theta + \tan^4\theta}$. (Hint: Use the double angle formula for tangent twice.)
\vspace*{7cm}

\newpage

\question[10]
Solve the equation $\sin(3\theta) + \sin(\theta) = 0$ for all values of $\theta$ in the interval $[0, 2\pi]$.
\vspace*{5cm}

\question[10]
The height, $H$, in meters, of the tide in a certain harbor is modeled by the equation $H(t) = 10 + 4\sin(\frac{\pi}{6}t)$, where $t$ is the number of hours after midnight.
\begin{parts}
    \part What is the maximum and minimum height of the tide?
    \vspace*{2cm}
    \part At what times during a 24-hour day is the tide at its maximum height?
    \vspace*{3cm}
    \part For how many hours is the tide's height greater than 12 meters during a 24-hour period?
    \vspace*{4cm}
\end{parts}


\end{questions}

\end{document}

