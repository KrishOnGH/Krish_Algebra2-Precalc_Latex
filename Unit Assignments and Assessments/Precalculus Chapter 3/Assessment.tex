\documentclass[12pt]{exam}

\usepackage{amsmath}
\usepackage{amssymb}
\usepackage{amsfonts}
\usepackage{graphicx}
\usepackage{tikz}
\usetikzlibrary{arrows.meta}
\usepackage{lastpage}

% Header and Footer
\pagestyle{headandfoot}
\firstpageheader{Precalculus Mathematics for Calculus (Stewart)}{}{Assessment: Chapter 3}
\runningheader{Precalculus Mathematics for Calculus (Stewart)}{}{Assessment: Chapter 3}
\firstpagefooter{}{Page \thepage\ of \pageref{LastPage}}{}
\runningfooter{}{Page \thepage\ of \pageref{LastPage}}{}

% Title
\newcommand{\examtitle}{Chapter 3: Polynomial and Rational Functions}

% Instructions
\newcommand{\instructions}{
    \noindent\rule{\textwidth}{0.5pt}
    \begin{center}
    \textbf{Instructions:} Answer all questions to the best of your ability. Show all your work in the space provided for full credit. Unless otherwise stated, assume all variables represent real numbers and all functions are defined on their natural domains.
    \end{center}
    \noindent\rule{\textwidth}{0.5pt}
}

\begin{document}

\begin{center}
\textbf{\Large \examtitle} \\
\vspace{0.5cm}
\makebox[0.4\textwidth]{Name: \enspace\hrulefill}
\hspace{0.1\textwidth}
\makebox[0.4\textwidth]{Date: \enspace\hrulefill}
\end{center}

\instructions
\vspace{0.5cm}

\begin{questions}

\pointsinrightmargin

% Q1 — Polynomial graphs and features
\question[12]
Let $P(x)=-(x+2)^{3}(x-1)^{2}$.
\begin{parts}
    \part Determine the end behavior of $P$.
    \vspace*{1.8cm}
    \part Find all intercepts and state whether the graph crosses or only touches the $x$-axis at each zero.
    \vspace*{2.4cm}
    \part Sketch a clean graph, labeling intercepts and identifying the local behavior near each zero.
    \vspace*{2.0cm}
\end{parts}

\newpage

% Q2 — Division, Remainder/Factor Theorems
\question[12]
\begin{parts}
    \part For what value(s) of $k$ is $(x-2)$ a factor of $x^3+2kx^2+k^2x+k-4$?
    \vspace*{2.8cm}
    \part Let $f(x)=x^2+4x$. Solve $f(f(x))=f(x)$.
    \vspace*{3.0cm}
\end{parts}

\newpage

% Q3 — Real zeros and the Rational Root/IVT blend (concept + computation)
\question[12]
Let $p(x)=2x^{3}-x^{2}-5x+3$.
\begin{parts}
    \part Show that $p(0)>0$ and $p(1)<0$, and conclude there is a real zero in $(0,1)$.
    \vspace*{2.2cm}
    \part Use the Rational Root Theorem to list all rational candidates and explain why there is no rational root in $(0,1)$.
    \vspace*{2.6cm}
    \part Approximate this zero to two decimal places (justify your method).
    \vspace*{2.4cm}
\end{parts}

\newpage

% Q4 — Complex number arithmetic
\question[10]
Let $w=2+3i$ and $z=4-5i$.
\begin{parts}
    \part Compute $2w-3z$.
    \vspace*{1.6cm}
    \part Find $\displaystyle \frac{1}{w}$ in $a+bi$ form.
    \vspace*{1.8cm}
    \part Evaluate $\displaystyle \frac{2}{\overline{w}+z}$.
    \vspace*{1.8cm}
    \part Simplify $\displaystyle \frac{(1-i)^{4}}{(1+i)^{3}}$.
    \vspace*{1.8cm}
\end{parts}

\newpage

% Q5 — Fundamental Theorem of Algebra / Constructing polynomials from zeros
\question[12]
\begin{parts}
    \part Suppose one root of $x^{3}+ax^{2}-4x+b=0$ is $1+i$, where $a,b\in\mathbb{R}$. Find the other two roots and determine $a$ and $b$.
    \vspace*{3.2cm}
    \part If two factors of $x^{3}-t_{1}x+t_{2}$ are $x+2$ and $x-1$, find the roots of $x^{2}-t_{1}x+t_{2}$.
    \vspace*{2.6cm}
\end{parts}

\newpage

% Q6 — Rational functions: asymptotes, holes, graphing
\question[14]
Consider $s(x)=\dfrac{x^{3}-9x}{x^{2}-25}$.
\begin{parts}
    \part Determine the domain and all intercepts.
    \vspace*{2.2cm}
    \part Find all vertical asymptotes and the slant (oblique) asymptote.
    \vspace*{2.6cm}
    \part On a clean set of axes, sketch the graph, indicating sign on each interval of the domain and behavior near each asymptote.
    \vspace*{2.8cm}
\end{parts}

\newpage

% Q7 — Inequalities with polynomials/rationals
\question[12]
\begin{parts}
    \part Solve $\displaystyle \frac{x^{3}-9x}{x^{2}-25}\ge 0$. Express the solution in interval notation and mark excluded points.
    \vspace*{2.8cm}
    \part Solve $\displaystyle \frac{x^{2}(x-3)(x+2)}{(x-1)(x+2)}<0$. Carefully distinguish cancellations from domain restrictions.
    \vspace*{2.8cm}
\end{parts}

\end{questions}

\end{document}