\documentclass[12pt]{article}

\usepackage{amsmath}
\usepackage{amssymb}
\usepackage{amsfonts}
\usepackage{geometry}
\usepackage{fancyhdr}

% Page setup
\geometry{margin=1in}
\pagestyle{fancy}
\fancyhf{}
\fancyhead[L]{Integrated Algebra 2 and Precalculus}
\fancyhead[R]{Assignment: Chapter 4 of Algebra 2}
\fancyfoot[C]{Page \thepage}

\begin{document}

\begin{center}
\textbf{\Large Polynomial Factoring in Different Domains} \\
\vspace{0.5cm}
\makebox[0.4\textwidth]{Name: \enspace\hrulefill}
\hspace{0.1\textwidth}
\makebox[0.4\textwidth]{Date: \enspace\hrulefill}
\end{center}

\vspace{0.5cm}

\section{Introduction}

In our journey through algebra, we've learned to factor polynomials using various techniques. However, what many students don't realize is that \textit{how far} we can factor a polynomial depends on which number system we're working in. A polynomial that appears "completely factored" in one number system might factor further in a larger number system.

Today we'll explore polynomial factoring across different \textbf{integral domains} and \textbf{fields}, and learn about the fundamental theorem that governs polynomial factorization.

\section{Definitions and Number Systems}

\subsection{Integral Domains and Fields}

\textbf{Definition (Integral Domain):} An integral domain is a commutative ring with no zero divisors. In simpler terms, it's a number system where we can add, subtract, and multiply (with the usual properties), and where if $a \cdot b = 0$, then either $a = 0$ or $b = 0$.

\textbf{Definition (Field):} A field is an integral domain where every non-zero element has a multiplicative inverse. In other words, we can also divide by any non-zero element.

\subsection{Common Number Systems}

\begin{itemize}
\item $\mathbb{Z}$ = the integers $\{\ldots, -2, -1, 0, 1, 2, \ldots\}$ (integral domain, but not a field)
\item $\mathbb{Q}$ = the rational numbers $\left\{\frac{a}{b} : a, b \in \mathbb{Z}, b \neq 0\right\}$ (field)
\item $\mathbb{R}$ = the real numbers (field)
\item $\mathbb{C}$ = the complex numbers $\{a + bi : a, b \in \mathbb{R}, i^2 = -1\}$ (field)
\end{itemize}

\textbf{Inclusion Chain:} $\mathbb{Z} \subset \mathbb{Q} \subset \mathbb{R} \subset \mathbb{C}$

\subsection{Polynomial Rings}

When we write $F[x]$, we mean the ring of polynomials with coefficients from the system $F$. For example:
\begin{itemize}
\item $\mathbb{Z}[x]$ = polynomials with integer coefficients
\item $\mathbb{Q}[x]$ = polynomials with rational coefficients  
\item $\mathbb{R}[x]$ = polynomials with real coefficients
\item $\mathbb{C}[x]$ = polynomials with complex coefficients
\end{itemize}

\section{Irreducible vs. Reducible Polynomials}

\textbf{Definition:} A polynomial $p(x)$ of degree $\geq 1$ in $F[x]$ is called \textbf{irreducible over $F$} if it cannot be written as a product of two polynomials of positive degree with coefficients in $F$.

If $p(x)$ can be factored into polynomials of positive degree over $F$, then $p(x)$ is \textbf{reducible over $F$}.

\vspace{0.5cm}

\textbf{Key Insight:} A polynomial's irreducibility depends on which coefficient system we're using!

\section{Examples of Factoring Across Different Domains}

\subsection{Example 1: $x^2 - 2$}

\begin{itemize}
\item \textbf{Over $\mathbb{Z}[x]$:} $x^2 - 2$ is irreducible (cannot be factored with integer coefficients)
\item \textbf{Over $\mathbb{Q}[x]$:} $x^2 - 2$ is still irreducible (cannot be factored with rational coefficients)  
\item \textbf{Over $\mathbb{R}[x]$:} $x^2 - 2 = (x - \sqrt{2})(x + \sqrt{2})$ (now it factors!)
\item \textbf{Over $\mathbb{C}[x]$:} Same as over $\mathbb{R}[x]$ since $\sqrt{2} \in \mathbb{R} \subset \mathbb{C}$
\end{itemize}

\subsection{Example 2: $x^2 + 1$}

\begin{itemize}
\item \textbf{Over $\mathbb{Z}[x]$:} $x^2 + 1$ is irreducible
\item \textbf{Over $\mathbb{Q}[x]$:} $x^2 + 1$ is irreducible
\item \textbf{Over $\mathbb{R}[x]$:} $x^2 + 1$ is irreducible (no real square roots of $-1$)
\item \textbf{Over $\mathbb{C}[x]$:} $x^2 + 1 = (x - i)(x + i)$ (factors with complex numbers!)
\end{itemize}

\subsection{Example 3: $x^4 - 4$}

Let's trace this step by step:

\textbf{Over $\mathbb{Z}[x]$ and $\mathbb{Q}[x]$:}
$$x^4 - 4 = x^4 - 2^2 = (x^2)^2 - 2^2 = (x^2 - 2)(x^2 + 2)$$

Both factors are irreducible over $\mathbb{Q}$ (as we can verify), so this is the complete factorization.

\textbf{Over $\mathbb{R}[x]$:}
$$x^4 - 4 = (x^2 - 2)(x^2 + 2) = (x - \sqrt{2})(x + \sqrt{2})(x^2 + 2)$$

Now $x^2 - 2$ factors, but $x^2 + 2$ is still irreducible over $\mathbb{R}$.

\textbf{Over $\mathbb{C}[x]$:}
$$x^4 - 4 = (x - \sqrt{2})(x + \sqrt{2})(x - i\sqrt{2})(x + i\sqrt{2})$$

Now everything factors completely!

\section{The Fundamental Theorem of Algebra and Prime Factorization}

\subsection{Fundamental Theorem of Algebra}

\textbf{Theorem:} Every non-constant polynomial in $\mathbb{C}[x]$ has at least one root in $\mathbb{C}$.

\textbf{Corollary:} Every polynomial of degree $n \geq 1$ in $\mathbb{C}[x]$ factors completely into $n$ linear factors:
$$p(x) = a(x - r_1)(x - r_2)\cdots(x - r_n)$$
where $a$ is the leading coefficient and $r_1, r_2, \ldots, r_n$ are the roots (counting multiplicity).

\subsection{Prime Factorization for Polynomials}

Just as integers have unique prime factorization, polynomials have unique factorization into irreducible polynomials.

\textbf{Unique Factorization Theorem for Polynomials:} If $F$ is a field, then every non-constant polynomial in $F[x]$ can be written uniquely (up to order and units) as a product of irreducible polynomials in $F[x]$.

\textbf{Irreducible Polynomials in Different Fields:}
\begin{itemize}
\item Over $\mathbb{C}$: Only linear polynomials $(x - a)$ are irreducible
\item Over $\mathbb{R}$: Linear polynomials and quadratics with negative discriminant
\item Over $\mathbb{Q}$: More complex—requires tools like Eisenstein's criterion
\end{itemize}

\newpage

\section{Practice Problems}

\textbf{Part A: Basic Factoring Across Domains}

\textbf{1.} For each polynomial, factor it as completely as possible over the given domain:

\begin{enumerate}
\item[(a)] $x^2 - 3$ over $\mathbb{Q}[x]$, $\mathbb{R}[x]$, and $\mathbb{C}[x]$
\vspace{2cm}

\item[(b)] $x^2 + 4$ over $\mathbb{Q}[x]$, $\mathbb{R}[x]$, and $\mathbb{C}[x]$  
\vspace{2cm}

\item[(c)] $x^3 - 2$ over $\mathbb{Q}[x]$, $\mathbb{R}[x]$, and $\mathbb{C}[x]$
\vspace{2cm}
\end{enumerate}

\textbf{2.} Consider the polynomial $p(x) = x^4 - 5x^2 + 6$.

\begin{enumerate}
\item[(a)] Factor $p(x)$ completely over $\mathbb{Q}[x]$.
\vspace{2cm}

\item[(b)] Factor $p(x)$ completely over $\mathbb{R}[x]$.
\vspace{2cm}

\item[(c)] Factor $p(x)$ completely over $\mathbb{C}[x]$.
\vspace{2cm}
\end{enumerate}

\textbf{Part B: Understanding Irreducibility}

\textbf{3.} Determine whether each polynomial is irreducible over the given domain. Justify your answer.

\begin{enumerate}
\item[(a)] $x^2 + x + 1$ over $\mathbb{Q}[x]$
\vspace{2cm}

\item[(b)] $x^2 + x + 1$ over $\mathbb{R}[x]$
\vspace{2cm}

\item[(c)] $x^3 + x + 1$ over $\mathbb{Q}[x]$ (Hint: Check for rational roots)
\vspace{2cm}
\end{enumerate}

\textbf{4.} Show that $x^2 - 2$ is irreducible over $\mathbb{Q}[x]$ by proving that $\sqrt{2}$ is irrational.
\vspace{4cm}

\newpage

\section{Challenge Problems}

\textbf{5.} Consider the polynomial $f(x) = x^4 + x^3 + x^2 + x + 1$.

\begin{enumerate}
\item[(a)] Show that $f(x) = \frac{x^5 - 1}{x - 1}$ for $x \neq 1$.
\vspace{2cm}

\item[(b)] Use the fact that $x^5 - 1 = (x-1)(x^4 + x^3 + x^2 + x + 1)$ and the factorization of $x^5 - 1$ over $\mathbb{C}$ to factor $f(x)$ completely over $\mathbb{C}[x]$.
\vspace{3cm}

\item[(c)] What can you say about the irreducibility of $f(x)$ over $\mathbb{Q}[x]$? (This is the 5th cyclotomic polynomial)
\vspace{2cm}
\end{enumerate}

\newpage

\textbf{6.} \textbf{Eisenstein's Criterion:} A powerful tool for proving irreducibility over $\mathbb{Q}[x]$ is Eisenstein's criterion:

If $p(x) = a_n x^n + a_{n-1} x^{n-1} + \cdots + a_1 x + a_0$ with integer coefficients, and there exists a prime $p$ such that:
\begin{itemize}
\item $p$ does not divide $a_n$
\item $p$ divides $a_i$ for all $i < n$  
\item $p^2$ does not divide $a_0$
\end{itemize}
then $p(x)$ is irreducible over $\mathbb{Q}[x]$.

Use Eisenstein's criterion to prove that $x^3 + 2x + 2$ is irreducible over $\mathbb{Q}[x]$.
\vspace{3cm}



\end{document} 