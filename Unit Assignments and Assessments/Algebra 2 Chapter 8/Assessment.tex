\documentclass[12pt]{exam}

\usepackage{amsmath}
\usepackage{amssymb}
\usepackage{amsfonts}
\usepackage{graphicx}
\usepackage{tikz}
\usetikzlibrary{arrows.meta}

% Header and Footer
\pagestyle{headandfoot}
\firstpageheader{Integrated Algebra 2 and Precalculus}{}{Exam: Chapter 8 of Algebra 2}
\runningheader{Integrated Algebra 2 and Precalculus}{}{Exam: Chapter 8 of Algebra 2}
\firstpagefooter{}{Page \thepage\ of \numpages}{}
\runningfooter{}{Page \thepage\ of \numpages}{}

% Title
\newcommand{\examtitle}{Polynomial Division and Roots}

% Instructions
\newcommand{\instructions}{
    \noindent\rule{\textwidth}{0.5pt}
    \begin{center}
    \textbf{Instructions:} Answer all questions to the best of your ability. Show all your work in the space provided for full credit.
    \end{center}
    \noindent\rule{\textwidth}{0.5pt}
}

\begin{document}

\begin{center}
\textbf{\Large \examtitle} \\
\vspace{0.5cm}
\makebox[0.4\textwidth]{Name: \enspace\hrulefill}
\hspace{0.1\textwidth}
\makebox[0.4\textwidth]{Date: \enspace\hrulefill}
\end{center}

\instructions
\vspace{0.5cm}

\begin{questions}

\pointsinrightmargin

\question[8]
Find a constant $c$ such that there is no remainder when $x^3 + cx^2 + 4x - 21$ is divided by $x - 3$.

\textit{Hint: You may use the Remainder Theorem or polynomial long division.}
\vspace*{4cm}

\question[10]
Find the quotient and remainder when $x^4 - 23x^3 + 11x^2 + 14x + 20$ is divided by $x + 5$.

\textit{Hint: Consider using synthetic division for this problem.}
\vspace*{4cm}

\question[8]
Find the quotient and remainder when $x^4 + 3x^3 - x^2 + 7x - 1$ is divided by $2 - x$.

\textit{Hint: Rewrite the divisor in standard form first.}
\vspace*{4cm}

\newpage

\question[10]
Find all roots of the following polynomial:
$$g(y) = 12y^3 - 28y^2 - 9y + 10$$

\textit{Hint: Look for rational roots first using the Rational Root Theorem.}
\vspace*{5cm}

\question[12]
When $y^2 + my + 2$ is divided by $y - 1$, the quotient is $f(y)$ and the remainder is $R_1$. When $y^2 + my + 2$ is divided by $y + 1$, the quotient is $g(y)$ and the remainder is $R_2$. If $R_1 = R_2$, then find $m$.

\textit{Hint: Use the Remainder Theorem to find expressions for $R_1$ and $R_2$.}
\vspace*{4cm}

\newpage

\question[10]
Suppose $q(x)$ and $r(x)$ are the quotient and remainder, respectively, when the polynomial $f(x)$ is divided by the polynomial $d(x)$. Show that if $x = a$ is a root of $d(x)$, then $r(a) = f(a)$.

\textit{Hint: Use the division algorithm for polynomials.}
\vspace*{4cm}

\question[8]
Find all roots of each of the following polynomials:
\begin{parts}
    \part $f(x) = x^3 - 4x^2 - 11x + 30$
    \vspace*{3cm}
    
    \part $g(t) = t^4 + 5t^3 - 19t^2 - 65t + 150$
    \vspace*{3cm}
\end{parts}

\newpage

\question[10]
Find the remainder when $x^{100} - 4x^{50} + 5x + 6$ is divided by $x^3 - 2x^2 - x + 2$.

\textit{Hint: Can you factor the cubic? Try factoring $x^3$ out of the first two terms. Can you then factor further?}
\vspace*{5cm}

\question[8]
Suppose that $f(x)$ is a polynomial with integer coefficients such that $f(2) = 3$ and $f(7) = -7$. Show that $f(x)$ has no integer roots.

\textit{Hint: Note that 3 and $-7$ are both odd.}

\textit{Hint: Is it possible for $f(0)$ to be even?}
\vspace*{4cm}

\newpage

\question[8]
How can we quickly tell that $x - 1$ is a factor of $x^5 + 6x^4 - 7x^3 + 2x^2 - 2$ without performing the long division?

\textit{Hint: Use the Factor Theorem.}
\vspace*{3cm}

\question[10]
Find the quotient and remainder for the following polynomial division:
$$x^2 - 19x + 17 \text{ divided by } x + 7$$

\textit{Hint: Use polynomial long division or synthetic division.}
\vspace*{4cm}

\question[12]
Teresa divides $3x^4 + 2x^3 - 7x^2 + 4x - 1$ by $x + 2$ and gets a quotient of $3x^3 - 4x^2 + x + 2$ and a remainder of 5. How can Teresa quickly realize that she made a mistake without performing the division again, and without multiplying $x + 2$ by the quotient?

\textit{Hint: Use the Remainder Theorem to check her work.}
\vspace*{4cm}

\newpage

\question[8]
The polynomial $p(x) = 3x^3 - 20x^2 + kx + 12$ is divisible by $x - 3$ for some constant $k$. Factor $p(x)$ completely.

\textit{Hint: Use the Factor Theorem to find $k$ first, then factor completely.}
\vspace*{5cm}

\end{questions}

\end{document}
