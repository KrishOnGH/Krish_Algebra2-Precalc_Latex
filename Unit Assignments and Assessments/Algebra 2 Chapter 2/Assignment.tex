\documentclass{article}
\usepackage{amsmath}
\usepackage{amssymb}
\usepackage{geometry}
\usepackage{array}

\geometry{a4paper, margin=1in}
\setlength{\parindent}{0pt}

\begin{document}

\section*{Algebra 2 Chapter 2 Assignment (Focus on Proofs)}
\subsection*{Problems}

\begin{enumerate}
    \item[21.] Prove: If $b \neq 0$ and $d \neq 0$, then $\frac{a}{b} \cdot \frac{c}{d} = \frac{ac}{bd}$. \\
    \textit{(Hint: Use Exercise 20. We will assume Exercise 20 is the property that for $b \neq 0, d \neq 0$, it has been proven that $\frac{1}{b} \cdot \frac{1}{d} = \frac{1}{bd}$.)}
    
    \vspace{1cm}

    \item[22.] Prove: If $c \neq 0$ and $d \neq 0$, then $\frac{1}{\frac{c}{d}} = \frac{d}{c}$.
    
    \vspace{1cm}

    \item[23.] Prove: If $b \neq 0$, $c \neq 0$ and $d \neq 0$, then $\frac{a}{b} \div \frac{c}{d} = \frac{ad}{bc}$.
\end{enumerate}

\newpage

\section*{Solutions}

\subsection*{Proof for Exercise 21}
\begin{tabular}{l p{7cm}}
    \textbf{Statements} & \textbf{Reasons} \\
    \hline
    1. $b \neq 0$ and $d \neq 0$ & 1. Given \\
    2. $\frac{a}{b} = a \cdot \frac{1}{b}$ and $\frac{c}{d} = c \cdot \frac{1}{d}$ & 2. Definition of division \\
    3. $\frac{a}{b} \cdot \frac{c}{d} = \left(a \cdot \frac{1}{b}\right) \cdot \left(c \cdot \frac{1}{d}\right)$ & 3. Substitution \\
    4. $= (a \cdot c) \cdot \left(\frac{1}{b} \cdot \frac{1}{d}\right)$ & 4. Commutative and Associative properties of multiplication \\
    5. $= ac \cdot \frac{1}{bd}$ & 5. Assumed from Exercise 20 \\
    6. $= \frac{ac}{bd}$ & 6. Definition of division \\
    7. $\therefore \frac{a}{b} \cdot \frac{c}{d} = \frac{ac}{bd}$ & 7. Transitive property of equality \\
\end{tabular}

\vspace{1cm}

\subsection*{Proof for Exercise 22}
\begin{tabular}{l p{7cm}}
    \textbf{Statements} & \textbf{Reasons} \\
    \hline
    1. $c \neq 0$ and $d \neq 0$ & 1. Given \\
    2. $\frac{1}{\frac{c}{d}} = 1 \div \frac{c}{d}$ & 2. Definition of fraction bar as division \\
    3. The reciprocal of a nonzero number $x$ is $\frac{1}{x}$. The reciprocal of $\frac{c}{d}$ is $\frac{1}{\frac{c}{d}}$. & 3. Definition of a reciprocal. \\
    4. $\frac{c}{d} \cdot \frac{d}{c} = \frac{cd}{dc} = 1$ & 4. Proof from exercise 21. \\
    5. The reciprocal of $\frac{c}{d}$ is $\frac{d}{c}$ & 5. Definition of a reciprocal ($a \cdot b = 1$) \\
    6. $\frac{1}{\frac{c}{d}} = \frac{d}{c}$ & 6. Substitution \\
\end{tabular}

\vspace{1cm}

\subsection*{Proof for Exercise 23}
\begin{tabular}{l p{7cm}}
    \textbf{Statements} & \textbf{Reasons} \\
    \hline
    1. $b \neq 0$, $c \neq 0$, $d \neq 0$ & 1. Given \\
    2. $\frac{a}{b} \div \frac{c}{d} = \frac{a}{b} \cdot \frac{1}{\frac{c}{d}}$ & 2. Definition of division \\
    3. $\frac{1}{\frac{c}{d}} = \frac{d}{c}$ & 3. Result from Exercise 22 \\
    4. $\frac{a}{b} \div \frac{c}{d} = \frac{a}{b} \cdot \frac{d}{c}$ & 4. Substitution \\
    5. $= \frac{ad}{bc}$ & 5. Result from Exercise 21 \\
    6. $\therefore \frac{a}{b} \div \frac{c}{d} = \frac{ad}{bc}$ & 6. Transitive property of equality \\
\end{tabular}

\end{document}
