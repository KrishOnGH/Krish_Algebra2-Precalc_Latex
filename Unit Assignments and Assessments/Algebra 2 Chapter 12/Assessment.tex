\documentclass[12pt]{exam}

\usepackage{amsmath}
\usepackage{amssymb}
\usepackage{amsfonts}
\usepackage{graphicx}
\usepackage{tikz}
\usetikzlibrary{arrows.meta}

% Header and Footer
\pagestyle{headandfoot}
\firstpageheader{Integrated Algebra 2 and Precalculus}{}{Exam: Chapter 12 of Algebra 2}
\runningheader{Integrated Algebra 2 and Precalculus}{}{Exam: Chapter 12 of Algebra 2}
\firstpagefooter{}{Page \thepage\ of \numpages}{}
\runningfooter{}{Page \thepage\ of \numpages}{}

% Title
\newcommand{\examtitle}{Trigonometry}

% Instructions
\newcommand{\instructions}{
    \noindent\rule{\textwidth}{0.5pt}
    \begin{center}
    \textbf{Instructions:} Answer all questions to the best of your ability. Show all your work in the space provided for full credit.
    \end{center}
    \noindent\rule{\textwidth}{0.5pt}
}

\begin{document}

\begin{center}
\textbf{\Large \examtitle} \\
\vspace{0.5cm}
\makebox[0.4\textwidth]{Name: \enspace\hrulefill}
\hspace{0.1\textwidth}
\makebox[0.4\textwidth]{Date: \enspace\hrulefill}
\end{center}

\instructions
\vspace{0.5cm}

\begin{questions}

\pointsinrightmargin

\question[8]
In triangle $ABC$, we have $\angle B = 90^\circ$ and $\sin A = 5/7$. Find $\tan C$.
\vspace*{4cm}

\question[8]
In $\triangle PQR$, we have $\angle Q = 90^\circ$ and $\sin P = 1/4$. Find $\sin R$.
\vspace*{4cm}

\question[8]
If $\sin x = 3 \cos x$, then what is $(\sin x)(\cos x)$?
\vspace*{4cm}

\newpage

\question[12]
Evaluate each of the following:
\begin{parts}
    \item $\sin 135^\circ$
    \vspace*{3cm}
    \item $\cos(-120^\circ)$
    \vspace*{3cm}
    \item $\tan 300^\circ$
    \vspace*{3cm}
    \newpage
    \item $\sin 630^\circ$
    \vspace*{3cm}
    \item $\cos 315^\circ$
    \vspace*{3cm}
    \item $\tan 150^\circ$
    \vspace*{3cm}
\end{parts}

\newpage

\question[8]
Explain why $\cos(360^\circ + \theta) = \cos \theta$ for any angle $\theta$.
\vspace*{4cm}

\question[8]
Explain why $\cos(180^\circ - \theta) = -\cos \theta$ for any acute angle $\theta$.
\vspace*{4cm}

\newpage

\question[10]
Recall that $\sec x = \frac{1}{\cos x}$. Show that for any angle $x$ for which $\cos x \neq 0$, we have $\tan^2 x + 1 = \sec^2 x$.
\vspace*{5cm}

\question[12]
Show that in any triangle $ABC$, we have
\[ \frac{a}{\sin A} = \frac{b}{\sin B} = \frac{c}{\sin C} = 2R, \]
where $R$ is the circumradius of $\triangle ABC$. This is the Extended Law of Sines.
\vspace*{6cm}

\newpage

\question[16]
The \textbf{angle of elevation} is the angle above the horizontal at which a viewer must look to see an object that is higher than the viewer. Similarly, the \textbf{angle of depression} is the angle below the horizontal at which a viewer must look to see an object that is below the viewer.

\par\addvspace{\baselineskip}
Answer each of the following problems to the nearest foot.
\begin{parts}
    \item A surveyor measures the angle of elevation from her feet to the top of a building as $5^\circ$. The surveyor knows that the building is 500 feet tall. Assuming the ground is flat and level between the surveyor and the building, how far away is the surveyor from the building?
    \vspace*{5cm}

    \item I'm standing at the peak of a mountain that is 14,000 feet above sea level. The angle of depression from this peak to a nearby smaller peak is $4^\circ$. On my map, these two peaks are represented by points that are 1 inch apart. If each inch on my map represents 1.2 miles, and there are 5280 feet in a mile, then how many feet above sea level is the second peak?
    \vspace*{6cm}
\end{parts}

\end{questions}

\end{document}

