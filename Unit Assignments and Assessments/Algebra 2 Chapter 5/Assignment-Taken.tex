\documentclass[12pt]{article}

\usepackage{amsmath}
\usepackage{amssymb}
\usepackage{amsfonts}
\usepackage{geometry}
\usepackage{fancyhdr}
\usepackage{tikz}

% Page setup
\geometry{margin=1in}
\pagestyle{fancy}
\fancyhf{}
\fancyhead[L]{Integrated Algebra 2 and Precalculus}
\fancyhead[R]{Assignment: Chapter 5 of Algebra 2}
\fancyfoot[C]{Page \thepage}

\begin{document}

\begin{center}
\textbf{\Large Rational Functions: Features and Behavior Analysis} \\
\vspace{0.5cm}
\makebox[0.4\textwidth]{Name: \underline{Krish Arora \hspace{3.5cm}} \enspace}
\hspace{0.1\textwidth}
\makebox[0.4\textwidth]{Date: \underline{7/16/25 \hspace{3.5cm}} \enspace}
\end{center}

\vspace{0.5cm}

\section{Introduction}

Rational functions are among the most important functions in mathematics, appearing everywhere from physics to economics. A \textbf{rational function} is any function that can be written as the quotient of two polynomials:

$$f(x) = \frac{P(x)}{Q(x)}$$

where $P(x)$ and $Q(x)$ are polynomials and $Q(x) \neq 0$.

In this assignment, we'll explore the key features of rational functions and introduce a powerful tool from calculus—derivatives—to analyze where these functions are increasing and decreasing.

\section{Key Features of Rational Functions}

\subsection{Domain and Vertical Asymptotes}

The \textbf{domain} of a rational function consists of all real numbers except those that make the denominator zero.

\textbf{Vertical Asymptotes} occur at values of $x$ where the denominator equals zero but the numerator does not (after canceling common factors).

\textbf{Example:} For $f(x) = \frac{x^2 - 1}{x^2 - 4}$:
\begin{itemize}
\item Domain: $x \in \mathbb{R}, x \neq \pm 2$
\item Vertical asymptotes: $x = -2$ and $x = 2$
\end{itemize}

\subsection{Horizontal and Oblique Asymptotes}

\textbf{Horizontal Asymptotes} describe the end behavior of rational functions:

For $f(x) = \frac{a_n x^n + \cdots + a_0}{b_m x^m + \cdots + b_0}$:
\begin{itemize}
\item If $n < m$: horizontal asymptote at $y = 0$
\item If $n = m$: horizontal asymptote at $y = \frac{a_n}{b_m}$
\item If $n = m + 1$: oblique asymptote (no horizontal asymptote)
\item If $n > m + 1$: no horizontal or oblique asymptote
\end{itemize}

\textbf{Oblique Asymptotes} occur when the degree of the numerator is exactly one more than the degree of the denominator. Find it by polynomial long division.

\subsection{Holes (Removable Discontinuities)}

A \textbf{hole} occurs when both the numerator and denominator have a common factor that cancels out.

\textbf{Example:} $g(x) = \frac{x^2 - 1}{x - 1} = \frac{(x-1)(x+1)}{x-1} = x + 1$ for $x \neq 1$

There's a hole at $x = 1$, and the $y$-coordinate of the hole is $g(1) = 1 + 1 = 2$.

\subsection{Intercepts}

\begin{itemize}
\item \textbf{$x$-intercepts}: Set the numerator equal to zero and solve
\item \textbf{$y$-intercept}: Evaluate $f(0)$ (if $x = 0$ is in the domain)
\end{itemize}

\section{Introduction to Derivatives and Function Behavior}

\subsection{The Concept of a Derivative}

The \textbf{derivative} of a function $f(x)$ at a point gives us the \textit{instantaneous rate of change} of the function at that point. Geometrically, it's the slope of the tangent line to the curve.

We denote the derivative as $f'(x)$ or $\frac{df}{dx}$.

\subsection{Basic Derivative Rules}

Here are some fundamental rules you'll need:

\textbf{Power Rule:} If $f(x) = x^n$, then $f'(x) = nx^{n-1}$

\textbf{Constant Rule:} If $f(x) = c$ (constant), then $f'(x) = 0$

\textbf{Sum/Difference Rule:} $(f \pm g)' = f' \pm g'$

\textbf{Product Rule:} $(fg)' = f'g + fg'$

\textbf{Quotient Rule:} $\left(\frac{f}{g}\right)' = \frac{f'g - fg'}{g^2}$

\subsection{The Connection: Derivatives and Increasing/Decreasing Behavior}

Here's the key insight:
\begin{itemize}
\item If $f'(x) > 0$ on an interval, then $f(x)$ is \textbf{increasing} on that interval
\item If $f'(x) < 0$ on an interval, then $f(x)$ is \textbf{decreasing} on that interval
\item If $f'(x) = 0$ at a point, the function may have a \textbf{critical point} (local max, min, or inflection point)
\end{itemize}

\subsection{Example: Analyzing $f(x) = \frac{x^2 - 4}{x}$}

Let's find where this function is increasing and decreasing.

\textbf{Step 1:} Find the derivative using the quotient rule.
$$f(x) = \frac{x^2 - 4}{x} = x - \frac{4}{x}$$

Using the simpler form:
$$f'(x) = 1 - \frac{d}{dx}\left(\frac{4}{x}\right) = 1 - 4 \cdot \frac{d}{dx}(x^{-1}) = 1 - 4(-1)x^{-2} = 1 + \frac{4}{x^2}$$

\textbf{Step 2:} Analyze the sign of $f'(x)$.

Since $\frac{4}{x^2} > 0$ for all $x \neq 0$, we have:
$$f'(x) = 1 + \frac{4}{x^2} > 0 \text{ for all } x \neq 0$$

\textbf{Step 3:} Conclusion.

The function $f(x) = \frac{x^2 - 4}{x}$ is increasing on $(-\infty, 0)$ and on $(0, \infty)$.

Note: We exclude $x = 0$ from the domain since the original function is undefined there.

\newpage

\section{Practice Problems}

\textbf{Part A: Identifying Features of Rational Functions}

\textbf{1.} For each rational function, find:
\begin{itemize}
\item Domain
\item Vertical asymptotes
\item Horizontal or oblique asymptotes
\item Holes (if any)
\item $x$ and $y$ intercepts
\end{itemize}

\begin{enumerate}
\item[(a)] $f(x) = \frac{2x - 6}{x + 3}$
\\[8pt]
\begin{minipage}[t][4cm][t]{\linewidth}
    $\displaystyle$Domain: $x \in R, x \neq -3$
    \\[8pt] Vertical Asymptotes: $x=-3$
    \\[8pt] Horizontal/Oblique Asymptotes: $y=2$
    \\[8pt] Holes: None
    \\[8pt] Intercepts: $x=3$ and $y=-2$
\end{minipage}

\item[(b)] $g(x) = \frac{x^2 - 9}{x^2 - 2x - 3}$
\\[8pt]
\begin{minipage}[t][4cm][t]{\linewidth}
    $\displaystyle$Domain: $x \in R, x \notin \{-1, 3\}$
    \\[8pt] Vertical Asymptotes: $x \in \{-1, 3\}$
    \\[8pt] Horizontal/Oblique Asymptotes: $y=0$
    \\[8pt] Holes: $x=3$
    \\[8pt] Intercepts: $x=\pm3$ and $y=3$
\end{minipage}

\item[(c)] $h(x) = \frac{x^3 + 1}{x^2 - 1}$
\\[8pt]
\begin{minipage}[t][4cm][t]{\linewidth}
    $\displaystyle$Domain: $x \in R, x \neq \pm 1$
    \\[8pt] Vertical Asymptotes: $x = \pm 1$
    \\[8pt] Horizontal/Oblique Asymptotes: $y=x$
    \\[8pt] Holes: $x=-1$
    \\[8pt] Intercepts: $x=-1$ and $y=-1$
\end{minipage}
\end{enumerate}

\newpage

\textbf{2.} Find the oblique asymptote for $f(x) = \frac{2x^2 + x - 3}{x - 1}$ using polynomial long division.
\begin{minipage}[t][4cm][t]{\linewidth}
    $\displaystyle \frac{2x^2+x-3}{x-1} = x+3 (x \neq 1)$
\end{minipage}

\newpage

\textbf{Part B: Introduction to Derivatives}

\textbf{3.} Find the derivative of each function:

\begin{enumerate}
\item[(a)] $f(x) = 3x^4 - 2x^2 + 7$
\\[8pt]
\begin{minipage}[t][2cm][t]{\linewidth}
    $\displaystyle f'(x) = 12x^3-4x$
\end{minipage}

\item[(b)] $g(x) = \frac{1}{x^2} + \sqrt{x}$ (Hint: Rewrite using negative and fractional exponents)
\\[8pt]
\begin{minipage}[t][2cm][t]{\linewidth}
    $\displaystyle f'(x) = \frac{d}{dx} x^{-2} + x^{\frac{1}{2}} = -\frac{2}{x^{3}} + \frac{1}{2\sqrt{x}}$
\end{minipage}

\item[(c)] $h(x) = (x^2 + 1)(x - 3)$ (Use the product rule)
\\[8pt]
\begin{minipage}[t][3cm][t]{\linewidth}
    $\displaystyle f'(x) = (2x)(x-3) + (x^2+1)(1) = 2x^2-6x+x^2+1 = 3x^2-6x+1$
\end{minipage}

\item[(d)] $k(x) = \frac{x^2 + 1}{x - 2}$ (Use the quotient rule)
\\[8pt]
\begin{minipage}[t][3cm][t]{\linewidth}
    $\displaystyle f'(x) = \frac{(2x)(x-2) - (x^2+1)(1)}{x^2-4x+4} = \frac{2x^2-4x-x^2-1}{x^2-4x+4} = \frac{x^2-4x-1}{x^2-4x-4}$
\end{minipage}
\end{enumerate}

\textbf{Part C: Analyzing Increasing/Decreasing Behavior}

\textbf{4.} For each rational function:
\begin{itemize}
\item Find the derivative
\item Determine where the derivative is positive, negative, or zero
\item State the intervals where the function is increasing or decreasing
\end{itemize}

\begin{enumerate}
\item[(a)] $f(x) = \frac{x^2}{x + 1}$
\\[8pt]
\begin{minipage}[t][3cm][t]{\linewidth}
    $\displaystyle f'(x) = \frac{(2x)(x+1) - (x^2)(1)}{x^2+2x+1} = \frac{2x^2+2x-x^2}{x^2+2x+1} = \frac{x^2+2x}{x^2+2x+1}$
    \\[8pt] Zero: $x \in \{-2, 0\}$ ; Negative: $x \in (-2, 0)$ ; Positive: $x \in (-\infty, -2)$ or $x \in (0, \infty)$
    \\[8pt] Decreasing: $x \in (-2, 0)$ ; Increasing: $x \in (-\infty, -2)$ or $x \in (0, \infty)$
\end{minipage}

\item[(b)] $g(x) = \frac{x}{x^2 + 1}$
\\[8pt]
\begin{minipage}[t][3cm][t]{\linewidth}
    $\displaystyle f'(x) = \frac{(1)(x^2+1) - (x)(2x)}{x^4+2x^2+1} = \frac{x^2+1-2x^2}{x^4+2x^2+1} = \frac{-x^2+1}{x^4+2x^2+1}$
    \\[8pt] Zero: $x \in \{-1, 1\}$ ; Negative: $x \in (-\infty, -1)$ or $x \in (1, \infty)$ ; Positive: $x \in (-1, 1)$
    \\[8pt] Decreasing: $x \in (-\infty, -1)$ or $x \in (1, \infty)$ ; Increasing: $x \in (-1, 1)$
\end{minipage}
\end{enumerate}

\newpage

\textbf{5.} Consider the function $f(x) = \frac{x^2 - 1}{x}$.

\begin{enumerate}
\item[(a)] Rewrite $f(x)$ in the form $f(x) = x - \frac{1}{x}$.
\\[8pt]
\begin{minipage}[t][1cm][t]{\linewidth}
    $\displaystyle f(x) = x - \frac{1}{x}$
\end{minipage}

\item[(b)] Find $f'(x)$.
\\[8pt]
\begin{minipage}[t][2cm][t]{\linewidth}
    $\displaystyle f'(x) = 1 + x^{-2}$
\end{minipage}

\item[(c)] Solve $f'(x) = 0$ to find critical points.
\\[8pt]
\begin{minipage}[t][2cm][t]{\linewidth}
    $\displaystyle f'(x)$ is never 0.
\end{minipage}

\item[(d)] Determine the sign of $f'(x)$ on each interval determined by the critical points and vertical asymptotes.
\\[8pt]
\begin{minipage}[t][2cm][t]{\linewidth}
    $\displaystyle f'(x)$ is always positive.
\end{minipage}

\item[(e)] State where $f(x)$ is increasing and decreasing.
\\[8pt]
\begin{minipage}[t][2cm][t]{\linewidth}
    $\displaystyle f(x)$ is always increasing since its rate of change (derivative) is always positive.
\end{minipage}
\end{enumerate}

\end{document}