\documentclass[12pt]{exam}

\usepackage{amsmath}
\usepackage{amssymb}
\usepackage{amsfonts}
\usepackage{graphicx}

% Header and Footer
\pagestyle{headandfoot}
\firstpageheader{Integrated Algebra 2 and Precalculus}{}{Exam: Chapter 1 of Algebra 2}
\runningheader{Integrated Algebra 2 and Precalculus}{}{Exam: Chapter 1 of Algebra 2}
\firstpagefooter{}{Page \thepage\ of \numpages}{}
\runningfooter{}{Page \thepage\ of \numpages}{}

% Title
\newcommand{\examtitle}{Basic Concepts of Algebra}

% Instructions
\newcommand{\instructions}{
    \noindent\rule{\textwidth}{0.5pt}
    \begin{center}
    \textbf{Instructions:} Answer all questions to the best of your ability. Show all your work in the space provided for full credit.
    \end{center}
    \noindent\rule{\textwidth}{0.5pt}
}

\begin{document}

\begin{center}
\textbf{\Large \examtitle} \\
\vspace{0.5cm}
\makebox[0.4\textwidth]{Name: \enspace\hrulefill}
\hspace{0.1\textwidth}
\makebox[0.4\textwidth]{Date: \enspace\hrulefill}
\end{center}

\instructions
\vspace{0.5cm}

\begin{questions}

%--------------------------------------------------------------------------
% Section 1: Evaluating and Simplifying Expressions
%--------------------------------------------------------------------------

\pointsinrightmargin
\question[16]
Evaluate each of the following expressions when $a = -8$ and $b = \frac{1}{2}$.

\begin{parts}
    \part $ab + 2b + 3a$
    \fillwithdottedlines{2cm}
    
    \part $\dfrac{a}{b}$
    \fillwithdottedlines{2cm}

    \part $4(a^2b + b^2a)$
    \fillwithdottedlines{3cm}

    \part $(a - 2)\sqrt{-ab}$
    \fillwithdottedlines{3cm}
\end{parts}

\question[5]
Simplify the product as much as possible.
\[ \frac{2}{3a^2 - 6b} \cdot \frac{9a^3 - 18ab}{10a^2} \]
\fillwithdottedlines{4cm}

%--------------------------------------------------------------------------
% Section 2: Properties of Number Systems
%--------------------------------------------------------------------------

\question[12]
Decide whether each set is a field under the operations of addition and multiplication. If the set is not a field, name at least one field property that does not hold.

\begin{parts}
    \part The natural numbers ($\mathbb{N}$)
    \fillwithdottedlines{2.5cm}

    \part The integers ($\mathbb{Z}$)
    \fillwithdottedlines{2.5cm}

    \part The rational numbers ($\mathbb{Q}$)
    \fillwithdottedlines{2.5cm}

    \part The negative rational numbers
    \fillwithdottedlines{2.5cm}
\end{parts}

%--------------------------------------------------------------------------
% Section 3: Solving Equations
%--------------------------------------------------------------------------

\question[10]
Solve for the variable in each equation.

\begin{parts}
    \part Find all values of $y$ such that $\dfrac{3}{2 + \sqrt{y}} + \dfrac{4}{2 + \sqrt{y}} = 1$.
    \fillwithdottedlines{4cm}

    \part What values of $x$ satisfy $\dfrac{\sqrt{x+1} + \sqrt{x-1}}{\sqrt{x+1} - \sqrt{x-1}} = 3$?
    \fillwithdottedlines{5cm}
\end{parts}

%--------------------------------------------------------------------------
% Section 4: Problem Solving
%--------------------------------------------------------------------------

\question[8]
\begin{parts}
    \part Let $x$ be the middle integer of three consecutive integers. What is the sum of these three integers in terms of $x$?
    \fillwithdottedlines{3cm}
    
    \part The sum of 23 consecutive integers is 2323. What is the largest of the integers? (Hint: Use the result from the first part.)
    \fillwithdottedlines{4cm}
\end{parts}

\question[6]
A grocer wants to mix peanuts and cashews to produce 20 lb of mixed nuts worth \$6.20/lb. How many pounds of each kind of nut should she use if peanuts cost \$4.80/lb and cashews cost \$8.00/lb?
\fillwithdottedlines{6cm}


\end{questions}

\end{document}