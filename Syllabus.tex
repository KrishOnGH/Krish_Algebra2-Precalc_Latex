\documentclass[11pt, letterpaper]{article}

% --- PACKAGES ---
\usepackage[margin=1in]{geometry} % Set page margins
\usepackage{amsmath}              % For advanced math environments
\usepackage{amsfonts}             % For math fonts
\usepackage{amssymb}              % For math symbols
\usepackage{parskip}              % Adds space between paragraphs, no indent
\usepackage{tabularx}             % For better tables
\usepackage{booktabs}             % For professional-looking tables
\usepackage{hyperref}             % For clickable links (if any)
\hypersetup{
    colorlinks=true,
    linkcolor=blue,
    filecolor=magenta,      
    urlcolor=cyan,
    pdftitle={Precalculus Syllabus},
    pdfpagemode=FullScreen,
}

% --- BEGIN DOCUMENT ---
\begin{document}

% --- CUSTOM TITLE BLOCK ---
\vspace*{0.5cm} % Add some space at the top
\noindent
{\Large \textbf{Integrated Algebra 2 and Precalculus Study}} \hfill {\large Tutor Owen Payton}

% --- COURSE OVERVIEW ---
\section*{Course Overview}
This course provides a comprehensive study of the functions and concepts prerequisite to the study of calculus. We will begin by reinforcing foundational algebra skills before moving into a deep exploration of polynomial, rational, exponential, logarithmic, and trigonometric functions. The course will conclude with advanced topics including analytic geometry, sequences, series, and possibly an introduction to the concept of limits. Our study will be guided by topics from two primary texts.

% --- REQUIRED TEXTS ---
\section*{Required Texts}
\begin{enumerate}
    \item Brown, Richard G., et al. \textbf{\textit{Algebra and Trigonometry: Structure and Method, Book 2}}. Houghton Mifflin, 2000.
    \item Stewart, James, et al. \textbf{\textit{Precalculus: Mathematics for Calculus, 7th Edition}}. Cengage Learning, 2015.
\end{enumerate}

% --- COURSE OUTLINE ---
\section*{Schedule of Topics}
The following topics will be covered, organized by textbook and chapter.

% --- BOOK 1 TOPICS ---
\subsection*{Topics from \textit{Algebra and Trigonometry: Structure and Method}}
\begin{itemize}
    \item Chapter 1: Basic Concepts of Algebra
    \item Chapter 2: Inequalities and Proof
    \item Chapter 3: Linear Equations and Functions
    \item Chapter 4: Products and Factor of Polynomials
    \item Chapter 5: Rational Expressions
    \item Chapter 6: Irrational and Complex Numbers
    \item Chapter 7: Quadratic Equations and Functions
    \item Chapter 8: Variation and Polynomial Equations
    \item Chapter 9: Analytic Geometry
    \item Chapter 10: Exponential and Logarithmic Functions
    \item Chapter 11: Sequences and Series
    \item Chapter 12: Triangle Trigonometry
    \item Chapter 13: Trigonometric Graphs and Identities
\end{itemize}

% --- BOOK 2 TOPICS ---
\subsection*{Topics from \textit{Precalculus: Mathematics for Calculus}}
\begin{itemize}
    \item Chapter 1: Fundamentals
    \item Chapter 2: Functions
    \item Chapter 3: Polynomial and Rational Functions
    \item Chapter 4: Exponential and Logarithmic Functions
    \item Chapter 5: Trigonometric Functions: Unit Circle Approach
    \item Chapter 6: Trigonometric Functions: Right Triangle Approach
    \item Chapter 7: Analytic Trigonometry
    \item Chapter 8: Polar Coordinates
    \item Chapter 9: Vectors in Two and Three Dimensions
    \item Chapter 10: Systems of Equations and Inequalities
    \item Chapter 11: Conic Sections
    \item Chapter 12: Sequences and Series
    \item Chapter 13: Limits: A Preview of Calculus* (If time permits)
\end{itemize}

% --- GRADING POLICY ---
\section*{Grading Policy}
Student assessment will be based on tests for each chapter and a final exam for each textbook. Two separate grades will be issued: one for the material from the \textit{Algebra and Trigonometry} textbook and one for the material from the \textit{Precalculus} textbook.

The weighting for each grade is as follows:
\begin{itemize}
    \item \textbf{Chapter Tests:} 75\% of the total grade. A test will be administered after the completion of each chapter.
    \item \textbf{Final Exam:} 25\% of the total grade. A comprehensive final exam will be given for each book's material.
\end{itemize}

\newpage
% --- GRADE CALCULATION SHEET ---
\section*{Grade Calculation Sheet}

% --- BOOK 1 GRADE SHEET ---
\subsection*{\textit{Algebra and Trigonometry: Structure and Method}}
\subsubsection*{Chapter Tests (75\% of Grade)}
\begin{tabular}{|p{6cm}|p{3cm}|}
\hline
\textbf{Chapter Test} & \textbf{Score} \\
\hline
Chapter 1 & 96.4\% \\ \hline
Chapter 2 & 98\% \\ \hline
Chapter 3 & 99\% \\ \hline
Chapter 4 & 90\% \\ \hline
Chapter 5 & 94\% \\ \hline
Chapter 6 & 70\% \\ \hline
Chapter 7 & 94\% \\ \hline
Chapter 8 & 92\% \\ \hline
Chapter 9 & 100\% \\ \hline
Chapter 10 & 88\% \\ \hline
Chapter 11 & 100\% \\ \hline
Chapter 12 & 100\% \\ \hline
Chapter 13 & 97\% \\ \hline
\textbf{Test Average (\%)} & 94\% \\
\hline
\end{tabular}

\subsubsection*{Final Exam (25\% of Grade)}
\begin{tabular}{|p{6cm}|p{3cm}|}
\hline
\textbf{Final Exam} & \textbf{Score} \\
\hline
Final Exam Score (\%) & 96\% \\
\hline
\end{tabular}

\subsubsection*{Final Grade Calculation}
\fbox{\begin{minipage}{0.9\textwidth}
\vspace{0.2cm}
(Chapter Test Average 94\% $\times$ 0.75) + (Final Exam Score 96\% $\times$ 0.25) = \textbf{Final Grade} 94.5\%
\vspace{0.2cm}
\end{minipage}}

% --- BOOK 2 GRADE SHEET ---
\subsection*{\textit{Precalculus: Mathematics for Calculus}}
\subsubsection*{Chapter Tests (75\% of Grade)}
\begin{tabular}{|p{6cm}|p{3cm}|}
\hline
\textbf{Chapter Test} & \textbf{Score} \\
\hline
Chapter 1 & \\ \hline
Chapter 2 & \\ \hline
Chapter 3 & \\ \hline
Chapter 4 & \\ \hline
Chapter 5 & \\ \hline
Chapter 6 & \\ \hline
Chapter 7 & \\ \hline
Chapter 8 & \\ \hline
Chapter 9 & \\ \hline
Chapter 10 & \\ \hline
Chapter 11 & \\ \hline
Chapter 12 & \\ \hline
Chapter 13 & \\ \hline
\textbf{Test Average (\%)} & \\
\hline
\end{tabular}

\subsubsection*{Final Exam (25\% of Grade)}
\begin{tabular}{|p{6cm}|p{3cm}|}
\hline
\textbf{Final Exam} & \textbf{Score} \\
\hline
Final Exam Score (\%) & \\
\hline
\end{tabular}

\subsubsection*{Final Grade Calculation}
\fbox{\begin{minipage}{0.9\textwidth}
\vspace{0.2cm}
(Chapter Test Average \rule{2cm}{0.4pt} $\times$ 0.75) + (Final Exam Score \rule{2cm}{0.4pt} $\times$ 0.25) = \textbf{Final Grade} \rule{2cm}{0.4pt}
\vspace{0.2cm}
\end{minipage}}

\end{document}